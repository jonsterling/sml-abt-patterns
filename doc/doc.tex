\documentclass[11pt]{amsart}
\usepackage{setspace,url,hyperref}
\usepackage{amssymb,stmaryrd}
\usepackage{proof}
\usepackage[usenames,dvipsnames,svgnames,table]{xcolor}
\newtheorem{thm}{Theorem}[section]
\newtheorem{prop}[thm]{Proposition}
\newtheorem{lem}[thm]{Lemma}
\newtheorem{cor}[thm]{Corollary}
\theoremstyle{definition}
\newtheorem{definition}[thm]{Definition}
\newtheorem{example}[thm]{Example}
%\newtheorem{note}[thm]{Note}
\theoremstyle{remark}
\newtheorem{remark}[thm]{Remark}
\numberwithin{equation}{section}

\def\InputModeColorName{MidnightBlue}
\def\OutputModeColorName{Maroon}
\newcommand\InputMode[1]{{\color{\InputModeColorName}{#1}}}
\newcommand\OutputMode[1]{{\color{\OutputModeColorName}{#1}}}

\newcommand\HypJ[2]{#1\ (#2)}
\newcommand\GenJ[2]{\vert_{#1}\; #2}
\newcommand\JJ{\mathcal{J}}
\newcommand\IsVer[1]{\InputMode{#1}\ \mathit{ver}}
\newcommand\IsTrue[1]{\InputMode{#1}\ \mathit{true}}
\newcommand\IsProp[1]{\InputMode{#1}\ \mathit{prop}}
\newcommand\IsSet[1]{\InputMode{#1}\ \mathit{set}}
\newcommand\EqSet[2]{\InputMode{#1}=\InputMode{#2}\ \mathit{set}}
\newcommand\Member[2]{\InputMode{#1}\in \InputMode{#2}}
\newcommand\NotMember[2]{\InputMode{#1}\notin \InputMode{#2}}
\newcommand\EqMember[3]{\InputMode{#1}=\InputMode{#2}\in \InputMode{#3}}
\newcommand\Seq[2]{\InputMode{#1}\vdash #2}
\newcommand\SSeq[2]{\InputMode{#1}\vDash #2}
\newcommand\Eval[2]{\InputMode{#1}\Rightarrow \OutputMode{#2}}

\newcommand\Nat{\mathbb{N}}
\newcommand\True{\top}
\newcommand\False{\bot}
\newcommand\Imp[2]{#1\supset #2}
\newcommand\Conj[2]{#1\land #2}
\newcommand\Disj[2]{#1\lor #2}
\newcommand\Forall[3]{(\forall #2 \in #1)#3}
\newcommand\Exists[3]{(\exists #2 \in #1)#3}
\newcommand\It{\star}
\newcommand\Lam[2]{(\lambda #1) #2}
\newcommand\Pair[2]{\langle #1, #2 \rangle}
\newcommand\Inl[1]{\mathsf{inl}(#1)}
\newcommand\Inr[1]{\mathsf{inr}(#1)}

\newcommand\Wild{\_}

\newcommand\IsType[1]{\InputMode{#1}\ \mathit{type}}
\newcommand\IsValence[1]{\InputMode{#1}\ \mathit{valence}}
\newcommand\IsArity[1]{\InputMode{#1}\ \mathit{arity}}
\newcommand\IsCtx[1]{\InputMode{#1}\ \mathit{ctx}}
\newcommand\CtxAppend[3]{\InputMode{#1}\oplus\InputMode{#2}\leadsto\OutputMode{#3}}
\newcommand\CtxLookup[3]{\InputMode{#1}\ni\InputMode{#2}:\OutputMode{#3}}
\newcommand\Dom[1]{|#1|}

\newcommand\OPS{\mathcal{O}}
\newcommand\Arity{\mathfrak{A}}

\newcommand\ChkPat[4]{\InputMode{#1}\mid\OutputMode{#4}\Vdash\InputMode{#2}:\InputMode{#3}}
\newcommand\pv[1]{\mathfrak{#1}}
\newcommand\Abs[2]{[#1]#2}

\newcommand\jframe[1]{\framebox{\ensuremath{#1}}}


\begin{document}

\title{patterns for abts}

\author{Jonathan Sterling}
\address{}

\onehalfspacing

\maketitle

We will work with the very simple type systems of abstract binding trees, where
a type is a valence, or it is a vector of valences.

\begin{gather*}
  \infer{
    \IsValence{n}
  }{
    \Member{n}{\Nat}
  }
  \qquad
  \infer{
    \IsArity{(m_0;\dots;m_n)}
  }{
    \GenJ{i}{\HypJ{\Member{m_i}{\Nat}}{\InputMode{i} < \InputMode{n}}}
  }\\[6pt]
  \infer{
    \IsType{v}
  }{
    \IsValence{v}
  }
  \qquad
  \infer{
    \IsType{\vec{v}}
  }{
    \IsArity{\vec{v}}
  }
\end{gather*}

The type system is parameterized over a set of operators $\OPS$ with a
metaoperation $\Arity$ such that $\GenJ{\vartheta}{\HypJ{\IsArity{\Arity(\vartheta)}}{\Member{\vartheta}{\OPS}}}$.

\begin{gather*}
  \infer{
    \IsCtx{\cdot}
  }{
  }
  \qquad
  \infer{
    \IsCtx{\Psi, x:\tau}
  }{
    \IsCtx{\Psi} &
    \NotMember{x}{\Dom\Psi}
  }
\end{gather*}

The context concatenation judgment \jframe{\CtxAppend{\Psi}{\Psi'}{\Psi''}}
(presupposing $\IsCtx{\Psi}$, $\IsCtx{\Psi'}$) admits the postsupposition
$\IsCtx{\Psi'}$.

\begin{gather*}
  \infer{
    \CtxAppend{\Psi}{\cdot}{\Psi}
  }{
  }
  \qquad
  \infer{
    \CtxAppend{\Psi}{\Psi',x:\tau}{\Psi'',x:\tau}
  }{
    \CtxAppend{\Psi}{\Psi'}{\Psi''} &
    \NotMember{x}{\Dom{\Psi'}}
  }
\end{gather*}

The judgment \jframe{\ChkPat{\Psi}{p}{\tau}{\Lambda}} (presupposing
$\IsCtx{\Psi}$, $\IsType{\tau}$) expresses the checking of patterns
$\InputMode{p}$ at type $\InputMode{\tau}$ with respect to an intuitionistic
context $\InputMode{\Psi}$, synthesizing a linear context
$\OutputMode{\Lambda}$ (admitting the postsupposition $\IsCtx{\Lambda}$).

\begin{gather*}
  \infer[\text{nil}]{
    \ChkPat{\Psi}{(\cdot)}{(\cdot)}{\cdot}
  }{
  }
  \qquad
  \infer[\text{snoc}]{
    \ChkPat{\Psi}{(\vec{p};p)}{(\vec{v};v)}{\Lambda}
  }{
    \ChkPat{\Psi}{(\vec{p})}{(\vec{v})}{\Lambda_1} &
    \ChkPat{\Psi}{p}{v}{\Lambda_2} &
    \CtxAppend{\Lambda_1}{\Lambda_2}{\Lambda}
  }\\[6pt]
  \infer[\text{wildcard}]{
    \ChkPat{\Psi}{\Wild}{\tau}{\cdot}
  }{
  }
  \qquad
  \infer[\text{var}]{
    \ChkPat{\Psi}{x}{\tau}{\cdot}
  }{
    \CtxLookup{\Psi}{x}{\tau}
  }
  \qquad
  \infer[\text{pat-var}]{
    \ChkPat{\Psi}{\pv{p}}{\tau}{\pv{p}:\tau}
  }{
  }\\[6pt]
  \infer[\text{op}]{
    \ChkPat{\Psi}{\vartheta(\vec{p})}{0}{\Lambda}
  }{
    \Member{\vartheta}{\OPS} &
    \ChkPat{\Psi}{\vec{p}}{\Arity(\vartheta)}{\Lambda}
  }
  \qquad
  \infer[\text{abs}]{
    \ChkPat{\Psi}{\Abs{x}{p}}{\tau+1}{\Lambda}
  }{
    \ChkPat{\Psi,x:0}{p}{\tau}{\Lambda}
  }
\end{gather*}

The linear context $\Lambda$ specifies the \emph{pattern variables} which occur
in a pattern, and would have to be substituted during unification. On the other
hand, the intuitionistic context $\Psi$ contains variables which arise from the
binding structure of the language of patterns, and such variables may occur any
number of times; in a simpler system that did not contain abstraction
$\Abs{x}{p}$, the intuitionistic context would be unnecessary.

\end{document}
